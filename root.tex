% arara: lualatex: { shell: yes, synctex: yes }
% arara: biber
% arara: makeglossaries if found('aux', '@istfilename')
% arara: lualatex: { shell: yes, synctex: yes }
% arara: lualatex: { shell: yes, synctex: yes }
\documentclass[english]{scrbook}
\usepackage[english]{babel}
\usepackage{scrhack}
\usepackage[headsepline,automark,autooneside=false]{scrlayer-scrpage}
\usepackage{xcolor}
\usepackage{hyperref}
\usepackage{minted}
\usepackage{csquotes}
\usepackage{lastpage}
\usepackage{tcolorbox}
\usepackage{setspace}
\usepackage{amsmath,amssymb,amsthm}
\usepackage{booktabs}
\usepackage{array}
\usepackage{siunitx}
\usepackage{graphicx}
\usepackage{lmodern}
\usepackage{dirtree}
\usepackage[acronym]{glossaries-extra}
\usepackage[
backend=biber,
style=ieee,
defernumbers=true,
]{biblatex}
\usepackage{comment}
\usepackage{mdframed}
\usepackage{caption}

% Fix for missing dates
\usepackage{xpatch}
\xpatchbibdriver{online}
{\printtext[parens]{\usebibmacro{date}}}
{\iffieldundef{year}
	{}
	{\printtext[parens]{\usebibmacro{date}}}}
{}
{\typeout{There was an error patching biblatex-ieee (specifically, ieee.bbx's @online driver)}}

\onehalfspacing

% Setup hyperref
\hypersetup{
	colorlinks=true,
	linkcolor=blue,
	urlcolor=blue
}

% Setup headers
\clearpairofpagestyles
\ihead{Helm - A Quickstart Guide and Styleguide}
\ohead{\leftmark}
\cfoot*{\pagemark\ of \pageref*{LastPage}}
\setkomafont{pagenumber}{}
\renewcommand\chapterpagestyle{scrheadings}

% Setup nicer chapters
\newcommand\titlerule[1][1pt]{\rule{\textwidth}{#1}}

\RedeclareSectionCommand[innerskip=0pt]{chapter}

\renewcommand\chapterlineswithprefixformat[3]{%
	#2\nobreak%
	\Ifstr{#2}{}{}{\kern-\dp\strutbox}%
	\titlerule\par\nobreak%
	#3%
	\par\nobreak\titlerule%
}

\makeatletter
\renewcommand\chapterlinesformat[3]{%
	\titlerule\par\nobreak%
	\@hangfrom{#2}{#3}%
	\par\nobreak\titlerule%
}
\makeatother

% Setup glossary
\setabbreviationstyle[acronym]{long-short}
\usepackage[draft]{draftmark}

\addbibresource{bibliography.bib}

% Title Page
\title{Helm - A Quickstart Guide and Styleguide}
\subtitle{An opinionated starter to Helm Charts}
\author{Marcel Radzio}

% Glossary definitions
\newglossaryentry{k8sResources}{
	name={Kubernetes Resource},
	description={A Kubernetes Resource is a YAML document defining a thing inside of a Kubernetes Cluster. This for example can be a Pod}
}
\newacronym{k8s}{k8s}{Kubernetes}
\newacronym{oci}{OCI}{Open Container Initiative}
\newacronym{ux}{UX}{User Experience}
\newacronym{values}{values file}{values.yaml}
% Make it
\makeglossaries

\begin{document}
\maketitle
\tableofcontents

\chapter{What is Helm?}
Helm is a system for managing \Glspl{k8sResources} and being able to deliver a simple to deploy bundle to administrators using \Gls{k8s} similar to Docker Compose or Ansible.

It is one of the most used packaging systems in the \Gls{k8s} ecosystem and helps to provide a consistent \gls{ux} in the ecosystem.

\section{Helm as a packaging system}
Helm itself describes itself as \enquote{The package manager for Kubernetes}\cite{helmauthorsHelm}
This means it's main goal and design is centered around being able to package an application, deal with dependencies and versioning.

As a result of this Helm is a generic way to define an application deployment.
In doing that it shares similarities with systems like Ansible, nix and other packaging solutions in the Linux world. 
In the typical Docker illustration style the Helm deployment would be a delivery which contains a given set of Containers which are delivered together to build a single big thing.

\section{Helm as a template engine}
As part of being a package manager for Kubernetes it has to deal with definitions written in pure YAML.
Therefor it features a powerful template engine which works similar to how Ansible's template engine works.

Under the hood this is archived by using the Go template language\cite{helmauthorsTemplateFunctionsPipelines,thegoauthorsTemplatePackageText} as well as additional go functions provided to the users.

On top of the template language itself it also provides the option to print messages at the end of the installation or upgrade as  well as nesting charts, having libraries for charts and using a global value context.
These are needed to provide consistent helm charts across a company or set of related charts and to ensure a nice \gls{ux} for those deploying said chart.

\section{Features Helm gives us over pure \Glspl{k8sResources}}
Pure \Glspl{k8sResources} come with multiple downsides over helm or similar packaging systems for \Gls{k8s}.

\Gls{k8s} brings no update management or atomic deployment management itself.
This means that an admin would need to ensure that they manually deploy each thing every time, possibly in the correct order and then has to ensure that it did not break.
If despite all efforts it did break you now end up with no easy way to rollback.
Ideally one has a git to go from but this might not be a safe operation either and may lead to new issues.

Helm on the other side brings two major things to the table.
It provides a way to manage the version independent of the product version.
This is called the \emph{version} of the Helm package, while the version of the product is called \enquote{\emph{appVersion}} in Helm.
Helm also provides you with a history of deployments.
This list tracks if a deployment is in progress, successful, failed or rolled back.
Due to this list helm allows you to seamlessly \emph{rollback} to any prior deployment you made.\footnote{While a \emph{rollback} is possible it is not possible to jump forward afterwards}
Additionally to the manual rollback Helm has the feature of atomic updates, which automatically will rollback to the last successful deployment version in case of invalid templates, values\footnote{Only syntactic errors} or failures to rollout the deployment in the given timeout.

Beyond the managing of versions and updates Helm also provides a way to share and therefor depend on other charts.
To do this Helm has their own Chart Repo setup or can use regular \gls{oci} repositories.
Using these repositories it allows then to define dependencies on existing helm charts.
This is especially useful to fulfill the goal of having a single deployable bundle for an application.
For example an API server might want to bundle a PostgreSQL helm chart as a dependency.
Instead of then defining it themselves it can then depend on existing Charts and provide a consistent experience for consumers in the ecosystem.

\chapter{Structure of a Helm Package}

A Helm Package follows a strict structure of files and folders.
The typical structure looks like this\footnote{The test-connection.yaml file can be named anything.
This is the default name you get in the official template.}:
\dirtree{%
 .1 .
 .2 Chart.yaml.
 .2 README.md.
 .2 templates.
 .3 \_helpers.yaml.
 .3 NOTES.txt.
 .3 tests.
 .4 test-connection.yaml.
 .2  values.yaml.
}

\section{The \enquote{Chart.yaml}}
This is the file defining the metadata of the Helm Chart.
Important fields are the name, description, type, version and appVersion fields.
Most of these are self-explanatory.
Below the special fields will be explained.

\begin{figure}[!hp]
\centering
\begin{singlespace}
\begin{minted}[numbers=left, frame=lines,breaklines,breakanywhere,samepage=false]{yaml}
apiVersion: v2
name: matrix-neoboard-widget
description: A whiteboard widget for the Element messenger
type: application
version: 0.1.0
appVersion: "0.0.0"
home: https://github.com/nordeck/matrix-neoboard
\end{minted}
\end{singlespace}
\caption{A simple application Chart.yaml}\label{code:Chart.yaml}
\end{figure}
	
\subsection{The \emph{appVersion} field}

\subsection{The \emph{maintainers} field}
\subsection{Other available fields}


\section{The \enquote{values.yaml}}

\subsection{Images}

The core of every application in Kubernetes is the image used for deploying it.
This is being done in the \enquote{image} section of the  \gls{values}.

\begin{figure}[!hp]
\centering
\begin{singlespace}
\begin{minted}[numbers=left, frame=lines,breaklines,breakanywhere,samepage=false]{yaml}
# This sets the container image more information can be found here: https://kubernetes.io/docs/concepts/containers/images/
image:
  repository: nginx
  # This sets the pull policy for images.
  pullPolicy: IfNotPresent
  # Overrides the image tag whose default is the chart appVersion.
  tag: ""
# This is for the secretes for pulling an image from a private repository more information can be found here: https://kubernetes.io/docs/tasks/configure-pod-container/pull-image-private-registry/
imagePullSecrets: []
\end{minted}
\end{singlespace}
\caption{The \enquote{image} section of the \gls{values}}\label{code:image_section}
\end{figure}

Things to note here are the 3 fields it should contain:

\begin{enumerate}
	\item The repository which sets the image name. This would also include things like \enquote{ghcr.io} or other custom repositories used.
	\item The \enquote{pullPolicy} which defines how often it is pulled. By default this should be \enquote{IfNotPresent}. For latest tags it automatically defaults however to \enquote{Always} which, as the name says, will always pull the image when a pod is started.
	\item The \enquote{tag} field defines the value after the colon in a docker image. This should stay as an empty string by default since it will be pulled from the chart's \enquote{appVersion} field usually. It is meant to allow a consumer to change this if they need to.
\end{enumerate}

Additionally there is the \enquote{imagePullSecrets} field which allows you to pull from private repositories. For more information on this take a look at \url{https://kubernetes.io/docs/tasks/configure-pod-container/pull-image-private-registry/}

\subsection{Service Account}

\begin{figure}[!hp]
\centering
\begin{singlespace}
\begin{minted}[numbers=left, frame=lines,breaklines,breakanywhere,samepage=false]{yaml}
#This section builds out the service account more information can be found here: https://kubernetes.io/docs/concepts/security/service-accounts/
serviceAccount:
  # Specifies whether a service account should be created
  create: true
  # Automatically mount a ServiceAccount's API credentials?
  automount: true
  # Annotations to add to the service account
  annotations: {}
  # The name of the service account to use.
  # If not set and create is true, a name is generated using the fullname template
  name: ""
\end{minted}
\end{singlespace}
\caption{The \enquote{serviceAccount} section of the \gls{values}}\label{code:service_account_section}
\end{figure}

Service Accounts are required for accessing the resources of the \gls{k8s} Cluster itself.
They are scoped accounts to the cluster and require most likely more setup in the templates to actually be useful.
They are commonly used by operators or similar things which listen to or write to resources in the cluster.

\subsection{\enquote{service} and \enquote{ingress}}

\begin{figure}[!hp]
\centering
\begin{singlespace}
\begin{minted}[numbers=left, frame=lines,breaklines,breakanywhere,samepage=false]{yaml}
service:
  # This sets the service type more information can be found here: https://kubernetes.io/docs/concepts/services-networking/service/#publishing-services-service-types
  type: ClusterIP
  # This sets the ports more information can be found here: https://kubernetes.io/docs/concepts/services-networking/service/#field-spec-ports
  port: 80

# This block is for setting up the ingress for more information can be found here: https://kubernetes.io/docs/concepts/services-networking/ingress/
ingress:
  enabled: false
  className: ""
  annotations: {}
    # kubernetes.io/ingress.class: nginx
    # kubernetes.io/tls-acme: "true"
  hosts:
    - host: chart-example.local
      paths:
        - path: /
          pathType: ImplementationSpecific
  tls: []
    #  - secretName: chart-example-tls
    #    hosts:
    #      - chart-example.local
\end{minted}
\end{singlespace}
\caption{The \enquote{service} section and the \enquote{ingress} section of the \gls{values}}\label{code:service_and_ingress_section}
\end{figure}

\begingroup
\centering
\label{figure:values.yaml}
\begin{singlespace}
\begin{minted}[numbers=left, frame=lines,breaklines,breakanywhere,samepage=false]{yaml}
# This will set the replicaset count more information can be found here: https://kubernetes.io/docs/concepts/workloads/controllers/replicaset/
replicaCount: 1

# This is to override the chart name.
nameOverride: ""
fullnameOverride: ""

# This is for setting Kubernetes Annotations to a Pod.
# For more information checkout: https://kubernetes.io/docs/concepts/overview/working-with-objects/annotations/ 
podAnnotations: {}
# This is for setting Kubernetes Labels to a Pod.
# For more information checkout: https://kubernetes.io/docs/concepts/overview/working-with-objects/labels/
podLabels: {}

podSecurityContext: {}
  # fsGroup: 2000
  
securityContext: {}
  # capabilities:
  #   drop:
  #   - ALL
  # readOnlyRootFilesystem: true
  # runAsNonRoot: true
  # runAsUser: 1000
  # This is for setting up a service more information can be found here: https://kubernetes.io/docs/concepts/services-networking/service/
  
service:
  # This sets the service type more information can be found here: https://kubernetes.io/docs/concepts/services-networking/service/#publishing-services-service-types
  type: ClusterIP
  # This sets the ports more information can be found here: https://kubernetes.io/docs/concepts/services-networking/service/#field-spec-ports
  port: 80
  
# This block is for setting up the ingress for more information can be found here: https://kubernetes.io/docs/concepts/services-networking/ingress/
ingress:
  enabled: false
  className: ""
  annotations: {}
    # kubernetes.io/ingress.class: nginx
    # kubernetes.io/tls-acme: "true"
  hosts:
    - host: chart-example.local
      paths:
        - path: /
          pathType: ImplementationSpecific
  tls: []
  #  - secretName: chart-example-tls
  #    hosts:
  #      - chart-example.local
  
resources: {}
  # We usually recommend not to specify default resources and to leave this as a conscious
  # choice for the user. This also increases chances charts run on environments with little
  # resources, such as Minikube. If you do want to specify resources, uncomment the following
  # lines, adjust them as necessary, and remove the curly braces after 'resources:'.
  # limits:
  #   cpu: 100m
  #   memory: 128Mi
  # requests:
  #   cpu: 100m
  #   memory: 128Mi
  
# This is to setup the liveness and readiness probes more information can be found here: https://kubernetes.io/docs/tasks/configure-pod-container/configure-liveness-readiness-startup-probes/
livenessProbe:
  httpGet:
    path: /
    port: http
readinessProbe:
  httpGet:
    path: /
    port: http
    
#This section is for setting up autoscaling more information can be found here: https://kubernetes.io/docs/concepts/workloads/autoscaling/
autoscaling:
  enabled: false
  minReplicas: 1
  maxReplicas: 100
  targetCPUUtilizationPercentage: 80
  # targetMemoryUtilizationPercentage: 80
  
# Additional volumes on the output Deployment definition.
volumes: []
# - name: foo
#   secret:
#     secretName: mysecret
#     optional: false
# Additional volumeMounts on the output Deployment definition.

volumeMounts: []
# - name: foo
#   mountPath: "/etc/foo"
#   readOnly: true

nodeSelector: {}

tolerations: []

affinity: {}
\end{minted}
\end{singlespace}
\captionof{figure}{A simple application values.yaml}\label{code:values.yaml}
\endgroup

\section{The \enquote{NOTES.txt}}
\begin{figure}[!hp]
\centering
\begin{singlespace}
\begin{minted}[numbers=left, frame=lines,breaklines,breakanywhere,samepage=false]{text}
apiVersion: v2
name: matrix-neoboard-widget
description: A whiteboard widget for the Element messenger
type: application
version: 0.1.0
appVersion: "0.0.0"
home: https://github.com/nordeck/matrix-neoboard
\end{minted}
\end{singlespace}
\caption{A simple NOTES.txt}\label{code:NOTES.txt}
\end{figure}
%%%%%%%%%%%%%%%%%%%%%%%%
\cleardoublepage
\appendix
\printglossaries

\nocite{*}
\printbibliography[heading=bibintoc,title={Sources}]

\listoffigures
\end{document} 
