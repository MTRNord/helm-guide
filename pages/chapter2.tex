A Helm Package follows a strict structure of files and folders.
The typical structure looks like this\footnote{The test-connection.yaml file can be named anything.
	This is the default name you get in the official template.}:
\dirtree{%
 .1 .
 .2 Chart.yaml.
 .2 README.md.
 .2 templates.
 .3 \_helpers.yaml.
 .3 NOTES.txt.
 .3 tests.
 .4 test-connection.yaml.
 .2  values.yaml.
}
\clearpage

\section{The \enquote{Chart.yaml}}
This is the file defining the metadata of the Helm Chart.
Important fields are the name, description, type, version and appVersion fields.
Most of these are self-explanatory.
Below the special fields will be explained.

\begin{figure}[h]
\begin{minted}[numbers=left, frame=lines,breaklines,breakanywhere,samepage=false]{yaml}
apiVersion: v2
name: matrix-neoboard-widget
description: A whiteboard widget for the Element messenger
type: application
version: 0.1.0
appVersion: "0.0.0"
home: https://github.com/nordeck/matrix-neoboard
\end{minted}
\caption{A simple application Chart.yaml}\label{code:Chart.yaml}
\end{figure}
\clearpage

\subsection{The \emph{appVersion} field}
\lipsum[2-4] 
\subsection{The \emph{maintainers} field}
\lipsum[2-4] 
\subsection{Other available fields}
\lipsum[2-4] 

\section{The \enquote{values.yaml}}

\subsection{Images}

The core of every application in Kubernetes is the image used for deploying it.
This is being done in the \enquote{image} section of the  \gls{values}.

\begin{figure}[h]
\begin{minted}[numbers=left, frame=lines,breaklines,breakanywhere,samepage=false]{yaml}
# This sets the container image more information can be found here: https://kubernetes.io/docs/concepts/containers/images/
image:
  repository: nginx
  # This sets the pull policy for images.
  pullPolicy: IfNotPresent
  # Overrides the image tag whose default is the chart appVersion.
  tag: ""
# This is for the secretes for pulling an image from a private repository more information can be found here: https://kubernetes.io/docs/tasks/configure-pod-container/pull-image-private-registry/
imagePullSecrets: []
\end{minted}
\caption{The \enquote{image} section of the \gls{values}}\label{code:image_section}
\end{figure}

Things to note here are the 3 fields it should contain:

\begin{enumerate}
	\item{
		The repository which sets the image name.
		This would also include things like \enquote{ghcr.io} or other custom repositories used.
	}
	\item{
		The \enquote{pullPolicy} which defines how often it is pulled
		By default this should be \enquote{IfNotPresent}.
		For latest tags it automatically defaults however to \enquote{Always} which, as the name says, will always pull the image when a pod is started.
	}
	\item{
		The \enquote{tag} field defines the value after the colon in a docker image.
		This should stay as an empty string by default since it will be pulled from the chart's \enquote{appVersion} field usually.
		It is meant to allow a consumer to change this if they need to.
	}
\end{enumerate}

Additionally there is the \enquote{imagePullSecrets} field which allows you to pull from private repositories. For more information on this take a look at \url{https://kubernetes.io/docs/tasks/configure-pod-container/pull-image-private-registry/}
\clearpage

\subsection{Service Account}

\begin{figure}[h]
\begin{minted}[numbers=left, frame=lines,breaklines,breakanywhere,samepage=false]{yaml}
#This section builds out the service account more information can be found here: https://kubernetes.io/docs/concepts/security/service-accounts/
serviceAccount:
  # Specifies whether a service account should be created
  create: true
  # Automatically mount a ServiceAccount's API credentials?
  automount: true
  # Annotations to add to the service account
  annotations: {}
  # The name of the service account to use.
  # If not set and create is true, a name is generated using the fullname template
  name: ""
\end{minted}
\caption{The \enquote{serviceAccount} section of the \gls{values}}\label{code:service_account_section}
\end{figure}

Service Accounts are required for accessing the resources of the \gls{k8s} Cluster itself.
They are scoped accounts to the cluster and require most likely more setup in the templates to actually be useful.
They are commonly used by operators or similar things which listen to or write to resources in the cluster.
\clearpage

\subsection{Service and Ingress}
\lipsum[2-4]

\begin{figure}[h]
\begin{minted}[numbers=left, frame=lines,breaklines,breakanywhere,samepage=false]{yaml}
service:
  # This sets the service type more information can be found here: https://kubernetes.io/docs/concepts/services-networking/service/#publishing-services-service-types
  type: ClusterIP
  # This sets the ports more information can be found here: https://kubernetes.io/docs/concepts/services-networking/service/#field-spec-ports
  port: 80

# This block is for setting up the ingress for more information can be found here: https://kubernetes.io/docs/concepts/services-networking/ingress/
ingress:
  enabled: false
  className: ""
  annotations: {}
  # kubernetes.io/ingress.class: nginx
  # kubernetes.io/tls-acme: "true"
  hosts:
    - host: chart-example.local
      paths:
        - path: /
          pathType: ImplementationSpecific
  tls: []
  #  - secretName: chart-example-tls
  #    hosts:
  #      - chart-example.local
\end{minted}
\caption{The \enquote{service} section and the \enquote{ingress} section of the \gls{values}}\label{code:service_and_ingress_section}
\end{figure}
\clearpage

\subsection{Volumes}
\lipsum[2-4] 

\begin{figure}[h]
\begin{minted}[numbers=left, frame=lines,breaklines,breakanywhere,samepage=false]{yaml}
# Additional volumes on the output Deployment definition.
volumes: []
# - name: foo
#   secret:
#     secretName: mysecret
#     optional: false
# Additional volumeMounts on the output Deployment definition.

volumeMounts: []
# - name: foo
#   mountPath: "/etc/foo"
#   readOnly: true
\end{minted}
\caption{The \enquote{volumes} section and the \enquote{volumeMounts} section of the \gls{values}}\label{code:volumes_section}
\end{figure}
\clearpage

\subsection{Security Contexts}
\lipsum[2-4] 

\begin{figure}[h]
\begin{minted}[numbers=left, frame=lines,breaklines,breakanywhere,samepage=false]{yaml}
podSecurityContext: {}
# fsGroup: 2000

securityContext: {}
# capabilities:
#   drop:
#   - ALL
# readOnlyRootFilesystem: true
# runAsNonRoot: true
# runAsUser: 1000
# This is for setting up a service more information can be found here: https://kubernetes.io/docs/concepts/services-networking/service/
\end{minted}
\caption{The security context sections of the \gls{values}}\label{code:security_section}
\end{figure}
\clearpage

\subsection{Resources}
\lipsum[2-4] 

\begin{figure}[h]
\begin{minted}[numbers=left, frame=lines,breaklines,breakanywhere,samepage=false]{yaml}
resources: {}
# We usually recommend not to specify default resources and to leave this as a conscious
# choice for the user. This also increases chances charts run on environments with little
# resources, such as Minikube. If you do want to specify resources, uncomment the following
# lines, adjust them as necessary, and remove the curly braces after 'resources:'.
# limits:
#   cpu: 100m
#   memory: 128Mi
# requests:
#   cpu: 100m
#   memory: 128Mi
\end{minted}
\caption{The \enquote{resources} section of the \gls{values}}\label{code:resources_section}
\end{figure}
\clearpage

\subsection{Probes}

TODO: Explain why it probably makes no sense to keep in values.yaml

\lipsum[2-4] 
\clearpage

\subsection{Autoscaling}
\lipsum[2-4] 

\begin{figure}[h]
\begin{minted}[numbers=left, frame=lines,breaklines,breakanywhere,samepage=false]{yaml}
#This section is for setting up autoscaling more information can be found here: https://kubernetes.io/docs/concepts/workloads/autoscaling/
autoscaling:
  enabled: false
  minReplicas: 1
  maxReplicas: 100
  targetCPUUtilizationPercentage: 80
  # targetMemoryUtilizationPercentage: 80
\end{minted}
\caption{The \enquote{autoscaling} section of the \gls{values}}\label{code:autoscaling_section}
\end{figure}
\clearpage

\subsection{Misc}
\lipsum[2-4] 

\begin{figure}[h]
\begin{minted}[numbers=left, frame=lines,breaklines,breakanywhere,samepage=false]{yaml}
# This will set the replicaset count more information can be found here: https://kubernetes.io/docs/concepts/workloads/controllers/replicaset/
replicaCount: 1

# This is to override the chart name.
nameOverride: ""
fullnameOverride: ""

# This is for setting Kubernetes Annotations to a Pod.
# For more information checkout: https://kubernetes.io/docs/concepts/overview/working-with-objects/annotations/ 
podAnnotations: {}
# This is for setting Kubernetes Labels to a Pod.
# For more information checkout: https://kubernetes.io/docs/concepts/overview/working-with-objects/labels/
podLabels: {}

nodeSelector: {}

tolerations: []

affinity: {}
\end{minted}
\caption{Values which affect the pods or the deployment but are not in a specific group of things}\label{code:misc_values}
\end{figure}
\clearpage

\section{The \enquote{NOTES.txt}}
\lipsum[2-4] 

\begin{figure}[h]
\begin{minted}[numbers=left, frame=lines,breaklines,breakanywhere,samepage=false]{text}
TODO
\end{minted}
\caption{A simple NOTES.txt}\label{code:NOTES.txt}
\end{figure}
\clearpage